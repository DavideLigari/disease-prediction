\documentclass{class}
\usepackage{multicol}
\usepackage{float}
\usepackage{graphicx}
\usepackage{tabularx}
\usepackage{array}
\usepackage{caption}
\usepackage{enumitem} % Customize bullet lists
\usepackage{biblatex}

\addbibresource{References.bib}

% add path to images
\graphicspath{ {./images/} }

\title{Integrative Network Analysis: \\ Unveiling Symptom-Disease Interactions and Enhancing Predictive Models}

\githubrepo{https://github.com/DavideLigari01/financial-project}
\githubrepo{https://github.com/AndreaAlberti07/enhancing-disease-prediction}

\author{Andreoli C. • Ligari D. • Alberti A. • Scardovi M. }

\affil[1]{Department of Computer Engineering, Data ScienSickness predictionce, University of Pavia, Italy \newline\centering Course of Financial Data Science}
%% Corresponding author
\corrauthor{Author One}

%% Abbreviated author list for the running footer
\runningauthor{Alberti, Andreoli, Ligari, Scardovi}
\shortname{Integrative Network Analysis}
%\bibliography{References}

\begin{document}

\maketitle
\begin{abstract}

    We will write it once we have the results.

    \keywords{Graph theory • Features Engineering • Community detection • Null models • Random forest • MLP }
\end{abstract}
\begin{multicols}{2}
    \tableofcontents
    % \nocite{dizdar_dns_2021}

    % ------------------- START OF SECTIONS -------------------


    % ------------------- Introduction -------------------

    % Introduction to the medical problem and how it is currently addressed (some bibliography)
    
\section{Introduction}

\firstword{I}{n} the dynamic landscape of healthcare, understanding the intricate interplay between symptoms and diseases is paramount for effective diagnosis and prediction.
This report embarks on a comprehensive journey through the realms of network analysis,
leveraging both theoretical foundations and empirical data to unravel the complexities of symptom-disease interactions.
Our dual-fold objective is to provide a nuanced descriptive analysis of these interactions while identifying key features to bolster predictive models.\\
The foundation of this endeavor lies in an extensive review of existing literature,
drawing insights from seminal works on network theory and disease prediction.
By establishing a baseline through prior research, we pave the way for a deeper understanding
of the subject matter and ensure the relevance of our findings in the broader context of scientific inquiry.\\
Guided by insights gleaned from the literature, our exploration extends to the realm of data,
where we meticulously curate and analyze datasets of varying sizes.
Through a systematic process of exploratory data analysis and cleaning,
we prepare the groundwork for constructing meaningful networks that encapsulate the relationships between symptoms and diseases.\\
The heart of our analysis lies in the creation of intricate network structures,
employing bipartite models and non-weighted links to distill meaningful patterns.
We delve into a spectrum of network metrics, from fundamental measures like degree distribution and
clustering coefficients to more nuanced assessments of node importance and betweenness centrality.
Statistical significance is rigorously assessed through the lens of a null model, ensuring that our observations transcend mere chance.\\
Community detection algorithms further dissect the network, revealing hidden structures and relationships between diseases.
This not only enriches our understanding but also lays the groundwork for subsequent analyses.
As we traverse the terrain of network analysis, we introduce novel metrics inspired by the Hidalgo-Hausmann framework,
stratifying symptoms and diseases based on their predictive importance. These metrics, coupled with traditional measures like betweenness centrality,
contribute to the definition of features that fuel our predictive models.\\
With a robust foundation established, we transition to the realm of predictive modeling,
where our feature-rich approach promises to enhance the performance of established models.
Logistic regression, random forest, and multi-layer perceptron models are trained, tested, and validated,
with a keen eye on feature importance and model improvement strategies.\\
This report unfolds as a holistic exploration, weaving together theoretical frameworks, empirical analyses,
and predictive modeling into a cohesive narrative. As we traverse the intricate web of symptom-disease interactions,
our aim is not only to elucidate the underlying dynamics but also to pave the way for more accurate and insightful predictive models in the realm of healthcare.



    % ------------------- Dataset -------------------

    % Description of the dataset and its features (some plots and one hot encoding)
    \section{Dataset}

The dataset used for this project is obtained from Kaggle and is available at the
\href{https://www.kaggle.com/datasets/dhivyeshrk/diseases-and-symptoms-dataset?select=Final_Augmented_dataset_Diseases_and_Symptoms.csv}{following link}.
It comprises disease names along with the symptoms reported by the respective patients.

\noindent
\textbf{Overview:} The dataset encompasses 773 unique diseases and 377 symptoms, resulting in approximately 246,000 rows.
It was artificially generated while preserving Symptom Severity and Disease Occurrence Possibility.

\noindent
\textbf{Data Encoding:} To facilitate model training, the dataset utilizes one-hot encoding for each symptom, transforming
categorical symptom data into a binary format.

\noindent
\textbf{Class Imbalance:} The original dataset exhibited significant class imbalance, with some classes having only one
sample and others containing thousands. We addressed this problem using Oversampling and Undersampling techniques,
and the details are further elaborated in Section \ref{sec:methodology_ML}\ref{subsec:preliminary_data_preparation}.

\noindent
\textbf{Data Cleaning:} To allow consistent Oversampling, classes (diseases) with fewer than three symptoms were excluded
from the dataset, resulting in the removal of 25 classes. Additionally, diseases with no symptoms and symptoms with
no associated diseases were deleted as well.




    % ------------------- Goals -------------------

    % Description of the goals of the project:
    \section{Goals}
% - ML model to predict the disease
% - Network analysis to find the most important symptoms to reduce complexity, and enhance the available features
    % - ML model to predict the disease
    % - Network analysis to find the most important symptoms to reduce complexity, and enhance the available features


    % ------------------- Methodology and Results Network -------------------

    % Methodology used (very technical)
    
\subsection{Network Creation (Not Weighted - Bipartite)}

\subsection{L1 and L2 measures}

\subsection{Clustering}
To compute the average network clustering coefficient, as proposed by \Citeauthor{Watts_Strogatz_1998} \cite{Watts_Strogatz_1998}, it is possible to use the following
formula: 
\begin{equation}
    C = \frac{1}{n}\sum_{i = 1}^{n} C_i = \frac{1}{n}\sum_{i = 1}^{n} \frac{2e_i}{k_i(k_i-1)}
\end{equation}
where:
\begin{itemize}
    \item \textbf{n} is the total number of nodes in the network.
    \item \textbf{C\_i} represents the clustering coefficient of node \textbf{i}, which measures the degree to which the neighbors of node \textbf{i} are interconnected.
    \item \textbf{e\_i} stands for the number of edges that exist between the neighbors of node \textbf{i}.
    \item \textbf{k\_i} is the degree of node \textbf{i}, indicating the number of edges connected to node \textbf{i}.
\end{itemize}
Specifically, we used the version of clustering coefficient for bipartite graphs, redefined by \Citeauthor{Latapy_Magnien_Vecchio_2008} \cite{Latapy_Magnien_Vecchio_2008} and implemented
in the NetworkX function \texttt{nx.bipartite.average\_clustering}.


\subsection{Betweenness Centrality}

\subsection{Communities with Co-occurrence Matrix}



    % - network creation (adjacency matrix - not weighted - bipartite)
    % - hidalgo + significance
    % - communities with co-occurrence matrix
    % - clustering
    % - betweenness centrality

    
\subsection{Degree Distribution and Power Law}
% BEGIN: degree distribution and power law

% END: degree distribution and power law

\subsection{Most Important Symptoms/Diseases (4 Classes)}
% BEGIN: most important symptoms/diseases (4 classes)

% END: most important symptoms/diseases (4 classes)

\subsection{Communities}
% BEGIN: communities

% END: communities

\subsection{L2 Has No Sense, It's Right That Z-Score Low Value}
% BEGIN: L2 has no sense, it's right that z-score low value

% END: L2 has no sense, it's right that z-score low value

\subsection{Meaning of Z-Score}
% BEGIN: meaning of Z-score

% END: meaning of Z-score

\subsection{Betweenness Centrality}
% BEGIN: betweenness centrality

% END: betweenness centrality

    % - degree distribution and power law
    % - most important symptoms/diseases (4 classes) 
    % - communities
    % - L2 has no sense, it's right that z-score low value
    % - Meaning of Z-score
    % - betweenness centrality
    % - clustering coefficient

    % ------------------- Methodology and Results ML -------------------

    % How we built the ML model 
    \section{ML Model Methodology}
% - Face the unbalanced dataset problem
% - select best parameters for symptoms one hot only
% - select best parameters for combination of other features (best combination is chosen with random parameters looking at the accuracy)
% - Face the problem of normalization
% - train for each model the two version above with optimal parameters
% - pick the best model according to accuracy

    % - Face the unbalanced dataset problem
    % - select best parameters for symptoms one hot only
    % - select best parameters for combination of other features (best combination is chosen with random parameters looking at the accuracy)
    % - Face the problem of normalization
    % - train for each model the two version above with optimal parameters
    % - pick the best model according to accuracy

    % Results of the ML model (plots and tables)
    \section{ML Model Results}
\label{sec:results_ML}
In this section we present the results of the ML models we have trained. We then deeply inspect the best performing model, 
in order to understand its features and its performance.


% --------------- Model Selection ---------------

\subsection{Model Selection}
\label{subsec:results_ML_model_selection}

% MATTEO
% - Matrix plot of stepwise (which is results of algorithm 1) for each model --> to justify the chosen features

% CRISTIAN
% - Comparison for each model of the selected parameters --> to justify the chosen parameters

% ANDREA
% - Comparison of the three models with best combination of features and best parameters
	% - precision, recall, AUC, accuracy


% --------------- New Features Effect ---------------

\subsection{New Features Effect}
% ANDREA
% - compare the best symptoms one hot model with the best one with the new features



% --------------- Best Model Performance ---------------

\subsection{Best Model Analysis}
% DAVIDE
% - what are its features
% - confusion matrix computed on the 4 class diseases (HighL1 - HighL2, LowL1 - HighL2, ...)
% - Worst error
% - Most impactful symptoms



\subsection{Computational Complexity}
% MATTEO
% - apply the reduction technique based on symptoms importance (L1 and L2 combined in the 4 classes)
% - compare the performance of the reduced model with the original one
	% - precision, recall, AUC, accuracy
% - compare the times needed to train the two models


% CRISTIAN
% - 6 Modelli: 3 con solo sintomi, 3 con nuove features
% - - Salvati (joblib) 
% - - Tempo di training


    % - selected model performances (see which is the worst error or the most commonly misclassified diseases)
    % - Compute accuracy and PR AUC
    % - Compute the accuracy of the model dividing the data into the 4 classes (HighL1 - HighL2, LowL1 - HighL2, ...)
    % - only with one hot symptoms (with optimal parameters both for one hot and other features) 
    % - with new features 
    % - reduction in computational complexity

    % ---------------------- Conclusion -----------------------------

    \section{Conclusion}

Our study successfully integrates network analysis with machine learning to enhance disease prediction models in healthcare.
By analyzing symptom-disease networks using Symptom Influence (SI) and Disease Influence (DI) indices, we uncovered
critical patterns essential for accurate disease prediction. These indices revealed diverse symptom-disease associations,
guiding the selection of features for our models.\\
Logistic Regression emerged as the most effective model, balancing accuracy and complexity, particularly when augmented with network-based features.
This model demonstrated high accuracy and managed to capture complex patterns without significant overfitting.\\
A key achievement of our study is the effective balance between feature reduction and model performance.
Focusing on significant symptoms, we reduced training time substantially while maintaining high accuracy.
This approach is especially valuable in real-world applications where computational efficiency is crucial.\\
The study, however, also recognized challenges in disease prediction, as highlighted by the analysis of specific
diseases like bladder cancer and otitis media. These cases illustrated the intricacies involved in disease classification
and the necessity for continuous model refinement.
    \section{Limits and Future Works}
\label{sec:future_works}

In the pursuit of our final model, we navigated through a series of pivotal decisions, ranging from model selection, feature choices, 
and the intricate interplay of normalization techniques to hyperparameter tuning. These decisions, while steering us toward a robust model, 
come with inherent trade-offs, potentially leading to suboptimal outcomes. Here, we discuss some limitations in our approach and suggest 
avenues for future exploration.

\subsection{Limits}
\label{subsec:limits}

\begin{itemize}
\item \textbf{Feature Selection}: The selection of optimal features, as depicted in Figure \ref{fig:ML_operative_flow}, occurred 
before hyperparameter tuning. This sequential approach may result in the choice of an ostensibly optimal feature set, 
as both aspects are tightly linked.

\item \textbf{Hyperparameter Tuning}: The determination of the best hyperparameter combination relied on accuracy as 
the sole metric. While we employed a stratified cross-validation on a balanced dataset for a reliable accuracy estimate, 
a more comprehensive approach should encompass additional metrics such as precision, recall, and F1-score.

\item \textbf{Feature Reduction}: The feature reduction process evenly separated the four classes of symptoms and commenced 
retaining features from the class demonstrating the highest predictive power. This approach may yield suboptimal results, as a 
specific threshold might exist beyond which the predictive power of a feature diminishes. 
To better clarify this concept, let's consider the following example: suppose we have only two classes of symptoms, evenly distributed 
using the median as a threshold on the degree value. Suppose also that the predictive power of the features is the same for both classes.
In this case we cannot actually say that the degree doesn't impact the predictive power of the model. Indeed in the high degree class
we can have put lots of features with a degree not sufficiently high to become less relevant and these diseases end up
altering the result of the whole class, especially in a power law distribution context. 
A refined strategy involves employing a manual threshold for the degree value, identifying truly impactful features, 
potentially resulting in unbalanced classes.

\end{itemize}

\subsection{Future Work}

\begin{itemize}
\item \textbf{Symptoms Communities}: The features extracted from symptom communities were integrated into the model based on their 
inherent ability to capture relevant information. A potential enhancement involves leveraging this knowledge explicitly, using it as prior 
probability for the model. This entails favoring the most common diseases associated with the patient's symptoms and their communities.

\item \textbf{Multi-label Classification}: Our current approach treats diseases as independent entities. However, some diseases 
may be intricately connected. A prospective improvement entails treating diseases as a multi-label classification problem. 
For instance, the model could output the three most likely diseases instead of a singular one.

\item \textbf{Disease Complexity Analysis}: Our accuracy analysis extends to different classes of diseases based on their L1 and L2 values. 
A potential refinement involves a nuanced exploration of disease complexity, adjusting L1 and L2 thresholds to maximize accuracy 
differentials among disease classes. This approach would facilitate an in-depth analysis of diseases that pose higher prediction challenges.

\item \textbf{Rare Diseases}: As shown by the analysis, in many cases the model struggle to predict rare diseases. A valuable future
work could be analyzing the state of the art of rare diseases prediction in literature and try to apply it to our model.

\end{itemize}

    % - our approach in feature selection and parameters search is not optimal 
    % - list the possible issues (e.g. we are training model on diverse features with the optimal parameters for other features)
    % - exploit the knowledge of symptoms communities and their most pointed diseases, using it at prior probability for the model classification



    \clearpage
    \printbibliography

\end{multicols}
\end{document}