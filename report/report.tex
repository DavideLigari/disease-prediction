\documentclass{class}
\usepackage{multicol}
\usepackage{float}
\usepackage{graphicx}
\usepackage{tabularx}
\usepackage{array}
\usepackage{caption}
\usepackage{enumitem} % Customize bullet lists


\title{Integrative Network Analysis: \\ Unveiling Symptom-Disease Interactions and Enhancing Predictive Models}

\githubrepo{https://github.com/DavideLigari01/financial-project}
\githubrepo{https://github.com/AndreaAlberti07/enhancing-disease-prediction}

\author{Andreoli C. • Ligari D. • Alberti A. • Scardovi M. }



\affil[1]{Department of Computer Engineering, Data ScienSickness predictionce, University of Pavia, Italy \newline\centering Course of Financial Data Science}
%% Corresponding author
\corrauthor{Author One}

%% Abbreviated author list for the running footer
\runningauthor{Alberti, Andreoli, Ligari, Scardovi}
\shortname{Integrative Network Analysis}
\bibliography{References}

\begin{document}

\maketitle
\begin{abstract}
    We will write it once we have the results.

    \keywords{Graph theory • Features Engineering • Community detection • Null models • Random forest • MLP }
\end{abstract}
\begin{multicols}{2}
    \tableofcontents
    \clearpage
    % \nocite{dizdar_dns_2021}

    \nocite{dnsbaseddos}
    \nocite{taylor_four_2021}
    \nocite{Devi_2015}

    % ------------------- START OF SECTIONS -------------------

    % ------------------- Introduction -------------------
    % Introduction to the medical problem and how it is currently addressed (some bibliography)
    \input{./sections/Introduction.tex}

    % ------------------- Dataset -------------------
    % Description of the dataset and its features (some plots and one hot encoding)
    \input{./sections/Dataset.tex}

    % ------------------- Goals -------------------
    % Description of the goals of the project:
    % - ML model to predict the disease
    % - Network analysis to find the most important symptoms to reduce complexity, and enhance the available features
    \input{./sections/Goals.tex}

    % ------------------- Methodology -------------------
    % Methodology used (very technical)
    \input{./sections/Methodology_intro.tex}

    % How we built the network, how we computed metrics:
    \input{./sections/Methodology_network.tex}
    % - hidalgo + significance
    % - communities with adjacency matrix
    
    % How we built the ML model 
    \input{./sections/Methodology_ML.tex}
    % - Grid search (why we choose that specific greedy approach)
    % - Model selection
    % - features selection (2 approaches)
    
    % ------------------- Results -------------------
    % Results of the project
    \input{./sections/Results_intro.tex}

    % Results of the network analysis (plots and tables)
    \input{./sections/Results_network.tex}
    % - degree distribution
    % - clustering coefficient
    % - most important symptoms/diseases (4 classes) 
    % - communities
    
    % Results of the ML model (plots and tables)
    \input{./sections/Results_ML.tex} 
    % - selected model performances
    % - only with one hot symptoms 
    % - with new features 
    % - reduction in computational complexity)
    
    \clearpage
    \printbibliography

\end{multicols}
\end{document}